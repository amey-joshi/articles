\documentclass{article}
\usepackage{amsmath, amssymb, amsfonts, amsthm}
\usepackage{physics}
\title{High-temperature operation of GaN-based \\ vertical-cavity 
surface-emitting lasers \\ - A summary}
\date{17-Nov-2020}
\author{TBF}
\begin{document}
%\maketitle
This paper reports a study of the temperature dependence of the functioning of
a GaN-based vertical-cavity surface-emitting laser device. 
The device was fabricated by depositing undoped GaN layer of $2$-$\mu$m
thickness on the c-plane (0001) of a sapphire crystal. This was followed by
a $2$ $\mu$m thick layer of n-GaN with a dopant concentration of $4 \times
10^{18}$ cm$^{-3}$, a $9$ nm thick p-Al$_{0.2}$Ga$_{0.8}N$ layer containing
ten multiple quantum wells and finally a $170$ nm thick layer of p-type GaN
with a dopant concentration of $4 \times 10^{19}$ $cm^{-3}$. The last three
layers form a p-i-n type diode structure. The structures were grown using 
metal-organic chemical vapor epitaxy. This was followed by a silica layer of 
$45$ nm with current apertures of $10$ $\mu$m width. The current aperture was
covered with a $30$ nm thick Indium Tin oxide layer. Indium Tin oxide is an
optically transparent, electrically conducting crystal. This was followed by
creating distributed Bragg reflectors of $60$ $\mu$m diameter on either sides
of the device. The reflectors were Tantalum pentoxide crystals and they had a 
reflectivity greater than $0.99$.

The device begins lasing at a current density of $8.9$ kA/cm${^2}$. It
has a peak output power of $20$ $\mu$W. The V-I characteristics of the
device show that it is turned on at $3.1$ V; the current density starts to
rise above this threshold. The emission spectrum of the device shows a single 
longitudinal mode of $\lambda = 402.3$ nm with a line-width of $0.2$ nm. The
degree of polarization was measured to be $0.958$. The direction of polarization
varied from device to device. The authors reckon that this is because of the
circular aperture and the symmetry of the underlying crystal structure. The 
beam was circular with a very low divergence of $5\%$ full width at half maximum angle.

Below the threshold, the maximum emission intensity was at the center
of the aperture. However, beyond the threshold, when the lasing action begins,
the maximum emission intensity is shifted from the center. The authors 
attribute this shift to device inhomogeneity, local cavity length or transverse
optical confinement.

The lasing characteristics were examined at six different temperatures 
starting from room temperature to $350$ K with a quasi-continuous wave of $1\%$
duty cycle and a pulse width of $200$ $\mu$s. As the temperature rises the
current density at which lasing happens increases. The lasing threshold is
sharper at lower temperatures. The device achieves peak power output quickly 
after crossing the threshold. As temperature rises, the device needs additional
current density to reach the peak power output. At the highest temperature of
$350$ K, the device underwent a weak lasing transition and the peak power output
did not show a marked increase even at high current density. A higher
current density is needed at higher temperatures because of a greater leakage
current, non-radiative recombination of charge carriers and a temperature
dependent gain mode offset mismatch. This is also accompanied with a red-shift 
in the resonant wavelength at the rate of $0.012$ nm/K. The red shift happens
because of an increase in the refractive index of the material with temperature.
The gain peak also shifts at the rate of $0.034$ nm/K.

The authors also built a thermal model of the device and found that most of 
the heat is dissipated towards the p-side metal layer. The model also predicted
a saturation in the red-shift because of the increase in refractive index with
temperature is compensated by a decrease in the refractive index because of a
higher free carrier density of the leakage currents at higher temperature.

\end{document}
